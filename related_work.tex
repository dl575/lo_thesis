\chapter{Related Work}
\label{chap:related_work}

There are several existing projects which have looked into the problem of
resource management in the presence of timing requirements. One possible
solution is to use DVFS in conjunction with a real-time operating system (RTOS)
\cite{rtdvfs-systor12}. This is useful for hard real-time systems where timing
requirements are strict. However, it requires considerable programmer expertise
and effort in order to write applications in a form that is amenable to an RTOS
and to perform the detailed timing analysis required. For soft real-time
systems, such as the response-time requirements we are considering, it should
be possible to use a more lightweight approach to handle the timing
requirements. Zhu and Reddi \cite{zhu-hpca13} explored using DVFS and
heterogeneous cores in order to lower energy usage while still meeting website
loading times for a web browser. This was done by analyzing the relationship of
various HTML and CSS metrics to loading time. They showed a 9\% average
reduction in energy usage by using this information when selecting core and
DVFS operating point compared to an OS scheduler that uses only utilization
information.  However, their approach is specific to web browsing and does not
generalize to other applications. PACORA \cite{pacora-hotpar11} looks at
resource partitioning for datacenters while taking into account response-time
requirements. It can dynamically alter these allocations but these are done on
a course-grained level in order to detect changes in workload behavior rather
than on a fine-grained per-job level. Recent work has shown that timing on a
job-to-job basis can have large variations \cite{atlas-rtas13} and that
fine-grained resource allocations can have a high impact on energy efficiency
that is not captured by coarse-grained allocations \cite{padmanabha-micro13}.

Given the previous work in this domain, our approach is unique in targeting the
combination of several design objectives:
\begin{enumerate}
  \item \textbf{Generality: } The approach should work generally for any
  application with response-time requirements. It should not require
  application- or domain-specific knowledge.
  \item \textbf{Minimal programmer effort: } The additional programmer effort
  in order to enable this should be minimal. The only needed information from
  the programmer is the code segments of interest and their response-time
  requirement.
  \item \textbf{Fine-grained decision: } The resource allocation decision
  should be made on a per-job basis in order to handle job-to-job variation.
  For example, recent work has shown that video decoding time can vary greatly
  from frame-to-frame \cite{atlas-rtas13}. 
  \item \textbf{Predictive decision: } Our approach looks to predict resource
  allocations before a job is run. Since execution times can change largely
  from job to job, it is important to predict the execution time for the next
  job to run, rather than making decisions after poor response-time performance
  is seen.
\end{enumerate}


