\chapter{Introduction}
\label{chap:intro}

% Real-time systems are important
The last decade has seen an increased ubiquity of computers with the widespread
adoption of smartphones and tablets and the continued spread of embedded
cyber-physical systems. With this deep integration into our environments, it has
become important to consider the real-world interactions of these computers
rather than simply treating them as abstract computing machines. For example,
cyber-physical systems typically have real-time constraints in order to ensure
safe and correct physical interactions. Similarly, even mobile and desktop
systems, which are not traditionally considered real-time systems, are
inherently time-sensitive because of their interaction with users.
These real-time systems introduce new constraints and trade-offs for computer
systems design.

Real-time systems are typically specified as a set of tasks which have
real-time deadlines. Correct operation involves both correct computation of
outputs as well as finishing tasks before their deadlines.  Deadlines can
either be considered hard deadlines, where a missed deadline indicates a fatal
error, or soft deadlines, where a missed deadline results in poor performance
but is not fatal. Extensive work has been done in the software design and
analysis of real-time systems including real-time scheduling algorithms
\cite{rtschedulingsurvey-csur11}, real-time operating systems (RTOS)
\cite{rtossurvey-micro09, rtossurvey-icesc14}, and worst-case execution time
(WCET) analysis \cite{wcetsurvey-tecs08}.

% Thesis statement
In order to guarantee that deadlines are met, these systems are typically
designed conservatively so that even in the worst-case tasks will not exceed
their deadlines. As a result, in the average or typical case, tasks finish
before their deadline. This results in unused slack time between the task
finish and the deadline. One option to reclaim this unused time is to use it
for non-real-time (e.g., best effort) tasks. In this thesis, we explore the
software/hardware co-design of real-time systems in order to take advantage of
slack using modern hardware features. Specifically, we explore two angles for
improving computer systems. First, we explore the use of modern hardware
techniques for improving system security and reliability in the context of
real-time deadlines. Second, we take advantage of slack time to improve energy
efficiency by adjusting hardware resource allocations.

% Thesis statement
%On the other hand, traditional techniques proposed for improving
%hardware architectures are only evaluated for their average-case
%performance and are not designed to take into account the possibility of timing
%constraints. This thesis explore the development of hardware
%architectures that are aware of real-time requirements. Specifically, 
%we present solutions for applying modern hardware techniques for improving
%system security, reliability, and energy-efficiency to real-time systems.

\section{Secure and Reliable Hardware Architectures for Real-Time Systems}
\label{sec:intro.security}

% Run-time monitoring is useful
One potential use of extra time is for improving system security and
reliability by monitoring and checking run-time behavior.
Run-time monitoring techniques have been shown to be useful for improving the
reliability, security, and debugging capabilities of computer systems. For
example, Hardbound is a hardware-assisted technique to detect out-of-bound
memory accesses, which can cause undesired behavior or create a security
vulnerability if uncaught \cite{hardbound-asplos08}. Intel has recently
announced plans to support buffer overflow protection similar to Hardbound in
future architectures \cite{intel-mpx}. Similarly, run-time monitoring can
enable many other new security, reliability, and debugging capabilities such as
fine-grained memory protection \cite{mondrian-asplos02}, information flow
tracking \cite{dift-asplos04, testudo-micro08}, control flow integrity checking
\cite{hafix-dac15}, hardware error detection \cite{argus-micro07}, data-race
detection \cite{radish-isca12, cord-hpca06}, etc.  
However, today's parallel monitoring techniques cannot be easily applied
to critical real-time systems due to their lack of timing guarantees. Thus, we
have developed several techniques in order to enable run-time monitoring on
real-time systems.

% Parallel programmable monitoring
%There are several options on how to implement run-time monitoring.
%Implementing run-time monitoring in software using binary instrumentation or
%other similar methods introduces especially high overheads. For example,
%dynamic information flow tracking (DIFT) implemented in software suffers a 3.6x
%slowdown on average \cite{qin06-lift}. Implementing monitors in hardware
%greatly decreases the performance impact by performing monitoring in parallel
%to a program's execution. A dedicated hardware implementation of DIFT reduces
%average performance overheads to just 1.1\% \cite{suh-dift-asplos04}.  However,
%fixed hardware loses the programmability of software. A fixed hardware
%implementation cannot be updated and cannot change the type of run-time
%monitoring performed. Thus, recent studies have proposed using programmable
%parallel hardware, such as extra cores in a multi-core system or FPGA
%coprocessors, for monitoring \cite{chen08-lba, flexcore-micro10,
%harmoni-dsn12}. Our work in this paper is targeted at these programmable
%parallel hardware monitors.

% Monitoring WCET
Traditional real-time systems design requires estimation of the worst-case
execution time of tasks in order to guarantee schedulability. Thus, we first
developed a static analysis method to estimate the worst-case execution time
(WCET) impact of parallel run-time monitoring for real-time systems
(Chapter~\ref{chap:monitoring_wcet}). This allows run-time monitoring to be
enabled if the there is enough static slack in a real-time system's schedule.
Our analysis works by extending the traditional integer-linear programming
(ILP) formulation used for estimating WCET by adding new constraints in order
to model the impact of run-time monitoring.

% Monitoring Hard Drop
Slack exists not only in the real-time schedule (i.e., static slack) but also
at run-time due to tasks running faster than their deadline (i.e., dynamic
slack). In order to take advantage of this dynamic slack for security and
reliability, we developed a hardware architecture that selectively performs
monitoring while guaranteeing that deadlines are met
(Chapter~\ref{chap:monitoring_hard_drop}). The architecture dynamically enables
or disables monitoring in order to meet real-time deadlines.

% Monitoring Soft Drop
Finally, soft real-time and interactive systems do not require strict timing guarantees
but still look to achieve timely execution of programs. Thus, by building on
our work in applying monitoring to hard real-time systems, we created an
architecture that enables adjustable overheads by trading off monitoring coverage
(Chapter~\ref{chap:monitoring_dift_drop}). 

\section{Energy-Efficient Hardware Architectures for Real-Time Systems}
\label{sec:intro.energy}

For modern mobile and embedded systems, energy usage is a major concern due to
limited battery life. Energy is also an important concern for servers and
datacenters due to their significant contribution to operating costs.
Techniques, such as dynamic frequency and voltage scaling (DVFS) and
heterogeneous core mixes, have been developed to enable a trade off between
power and performance. These resource allocation techniques can be used to
reduce energy usage at the cost of increased execution time.

Traditional managers for these resource allocation techniques, such as DVFS,
work in a best effort manner. They operate under the assumption that better
performance is desired and only attempt to decrease resources when the
performance impact is minimal.  However, many applications or jobs have
response-time deadlines. Finishing faster than this response-time
requirement has no benefit. For example, responding to a user input faster than
human reaction time does not improve user experience. Similarly, decoding video
frames faster than the video frame rate has no added benefit. 

For these soft real-time systems, we can take advantage of this slack in order
to save energy without impacting user experience. In this thesis, we present a
method to predict the appropriate DVFS operating point for tasks in order to
meet deadlines with minimal energy (Chapter~\ref{chap:exec_time_prediction}).
This prediction needs to take into account the effect of inputs and program
state in order to account for dynamic variations in task execution time. We
show how this is possible through static analysis of task source code in order
to generate task-specific predictors for DVFS control.

\section{Organization}
The rest of the thesis is organized as follows.
Chapters~\ref{chap:monitoring_wcet}, \ref{chap:monitoring_hard_drop}, and
\ref{chap:monitoring_dift_drop} discuss our work on improving system security
and reliability through run-time monitoring on real-time systems. 
%Chapter~\ref{chap:monitoring_wcet} discusses the static analysis of run-time
%monitoring for hard real-time systems. Chapter~\ref{chap:monitoring_hard_drop}
%describes a dynamic scheme for implementing run-time monitoring on hard
%real-time systems. Chapter~\ref{chap:monitoring_dift_drop} extends this schemes
%to other applications with soft rather than hard deadlines.
Chapter~\ref{chap:exec_time_prediction} presents our work on creating
energy-efficient real-time systems. Finally, Chapter~\ref{chap:related_work}
discusses related work and Chapter~\ref{chap:conclusion} concludes the thesis.
