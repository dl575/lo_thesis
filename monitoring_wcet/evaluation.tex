\section{Evaluation}
\label{sec:monitoring_wcet.evaluation}

% Toolflow
\begin{figure}
  \begin{center}
    \includegraphics[width=3.5in]{monitoring_wcet/figs/toolflow.pdf}
    \caption{Toolflow for WCET estimation of parallel monitoring.}
    \label{fig:monitoring_wcet.evaluation.toolflow}
  \end{center}
\end{figure}

% Results
\begin{table*}[htb]
  \begin{center}
    \begin{scriptsize}
    \section{WCET Analysis}
\label{sec:monitoring_wcet.wcet}

\subsection{Implicit Path Enumeration}
\label{sec:monitoring_wcet.wcet.ipet}

Most of the WCET analysis techniques today rely on an ILP formulation that is
obtained from implicit path enumeration techniques \cite{li-ipet-dac95}.  In
this method, a program is converted to a control flow graph (CFG). From the
control flow graph, an ILP problem is formulated that seeks to maximize
\begin{align*}
  t = \sum_{B \in \mathcal{B}_{CFG}}{N_B \cdot c_{B,max}}
\end{align*} 
where $\mathcal{B}_{CFG}$ is the set of basic blocks in the control flow graph.
$N_{B}$ is the number of times block $B$ is executed and $c_{B,max}$ is the
maximum number of cycles to execute block $B$. The maximum value of $t$ is the
WCET of the task.  To account for the fact that only certain paths in the graph
will be executed, a set of constraints are placed on $N_{B}$. For example, on a
branch, only one of the branches will be taken on each execution of the block.
A variable can be assigned to each edge corresponding to the number of times
that edge is taken.  The number of times edges out of the block are taken must
equal the number of times the block is executed.  Similarly, the number of
times edges into the block are taken must equal the number of times the block
is executed. Various methods have been developed to create additional
constraints to convey other program behavior \cite{li-ipet-dac95,
wcetsurvey-tecs08}.

%% Why ILP?
Integer linear programming is an attractive optimization technique for this
problem because the solution found is a global optimum. In addition, many
aspects of program and architecture behavior can be described by adding
constraints to the ILP problem.  Several open source and commercial ILP solvers
exist which can solve the formulated ILP problem \cite{lpsolve, cplex}.  Thus,
in developing a method for estimating the WCET of parallel run-time monitoring,
we look to build upon this ILP framework.

%% Using worst case stalls in original WCET analysis to determine WCET with monitoring.
The IPET-based ILP formulation can be extended in a straightforward fashion to
incorporate run-time monitoring overheads if we have the maximum (worst-case)
monitoring stall cycles for each basic block by maximizing
\begin{align*}
  t = \sum_{B \in \mathcal{B}_{CFG}}{N_{B} \cdot (c_{B,max} + s_{B,max})}
\end{align*}
Here, $s_{B,max}$ represents the maximum number of cycles that block $B$ is
stalled due to monitoring. In this sense, the challenge in WCET analysis with
monitoring lies in determining $s_{B,max}$.  The rest of this section addresses
this problem.

\subsection{Sequential Monitoring Bound}
\label{sec:monitoring_wcet.wcet.sequential}

One way to determine a conservative bound on the worst-case monitoring stall
cycles is to consider sequential monitoring.  In sequential monitoring, the
monitoring task is run in-line with the main task on the same core rather than
in parallel. That is, after each instruction that would be forwarded, the
monitoring task is run on the main core before the main task resumes execution.
In this case, the WCET estimate can be obtained from a traditional method by
analyzing one program that contains both main and monitoring tasks.  The
resulting WCET can be considered as a simple bound for parallel monitoring
because it models the case where every forwarded instruction causes the main
core to stall.  However, this bound is extremely conservative as it does not
account for the FIFO buffering or the parallel execution of the monitoring
core. These features are critical to utilizing run-time monitoring techniques
while maintaining low performance overheads.

\subsection{FIFO Model}
\label{sec:monitoring_wcet.wcet.model}

To obtain tighter WCET bounds, we need to model the FIFO.  The main task can
continue its execution as long as a FIFO entry is available, but needs to stall
on a forwarded instruction if the FIFO is full. The WCET model needs to capture
the worst-case (maximum) number of entries in the FIFO at each forwarded
instruction and determine how many cycles the main task may be stalled due to
the FIFO being full. Here, we propose a mathematical model to express the load
in the FIFO and estimate the worst-case stalls.

In this approach, the original control flow graph must be transformed so that
each node contains at most one forwarded instruction which is located at the
end of the code sequence represented by the node.  This transformed graph is
called a {\em monitoring flow graph (MFG)}.  Intuitively, the analysis needs to
consider one forwarded instruction at a time in order to model the FIFO state
on each forwarded instruction and capture all potential stalls from monitoring. 

To model how full the FIFO is, we define the concept of \emph{monitoring load}.
The monitoring load is the number of cycles required for the monitoring core to
process all outstanding entries in the FIFO at a given point in time. The
monitoring load increases when a new instruction is forwarded by the main task,
and decreases as the monitoring core processes forwarded instructions. For
simplicity, the increase in monitoring load for any forwarded instruction is
conservatively assumed to be the worst-case (maximum) monitoring task execution
time among all possible forwarded instructions. This maximum, $t_{M, max}$, can
be obtained from the WCET analysis of the monitoring tasks.  We make this
simplification because it is difficult to model the FIFO mathematically at an
entry-by-entry level. With this simplification, each FIFO entry is identical
and so the monitoring load fully represents the state of the FIFO.  The
monitoring load cannot be negative and is upper-bounded by the maximum
monitoring load the FIFO can handle, $l_{max}$. The maximum monitoring load is
the number of FIFO entries, $n_F$, multiplied by the increase in monitoring
load for one forwarded instruction, $t_{M, max}$.

In order to find the worst-case monitoring stalls, we need to determine the
worst-case (maximum) monitoring load at the node boundaries in the MFG. For a
given node, $M$, in the MFG, we define $li_{M}$ as the monitoring load coming
into the node and $lo_{M}$ as the monitoring load exiting the node. The change
in monitoring load for the node is denoted by $\Delta l_{M}$. The maximum
$\Delta l_{M}$ can be calculated as the difference between the WCET of a
monitoring task that corresponds to $M$ and the minimum execution cycles of the
node, $c_{M,min}$: 
\begin{align*}
\Delta l_{M} = 
	\begin{cases}
	t_{M,max} - c_{M,min}, &\text{forwarded inst. } \in M \\
	- c_{M,min}, &\text{no forwarded inst. } \in M
	\end{cases}
\end{align*}
In order to ensure that the analysis is conservative in estimating the
worst-case (maximum) stalls, we use the best-case (minimum) execution time for
the main task here. 

Because the monitoring load is bounded by zero and the maximum load that
the FIFO can handle, $l_{max}$, the monitoring load coming out of a node is
\begin{align*}
	lo_{M} =& 
		\begin{cases}
			0, &li_{M} + \Delta l_{M} < 0 \\
			li_{M} + \Delta l_{M}, &0 \leq li_{M} + \Delta l_{M} \leq l_{max} \\
			l_{max}, &li_{M} + \Delta l_{M} > l_{max}
		\end{cases} \\
  l_{max} =& n_F \cdot t_{M,max}
\end{align*}

The worst-case monitoring load entering node $M$, $li_{M}$, is the largest of
the output monitoring loads among nodes with edges pointing to node $M$. Let
$\mathcal{M_{\text{prev}}}$ represent the set of nodes with edges pointing to
node $M$. Then,
\begin{align*}
	li_{M} = \max_{M_{prev} \in \mathcal{M}_{prev}}lo_{M_{prev}}
\end{align*}

The above equations describe the worst-case monitoring load at each node
boundary.  A monitoring stall occurs when a forwarded instruction is executed
but there is no empty entry in the FIFO buffer. In terms of monitoring load, if
a node would add monitoring load that would cause the resulting total load to
exceed $l_{max}$, then a monitoring stall occurs. The number of cycles stalled,
$s_{M}$, is the number of cycles that this total exceeds $l_{max}$. That is,
\begin{align*}
	s_{M} =
		\begin{cases}
			0, &li_{M} + \Delta l_{M} < l_{max} \\
			(li_{M} + \Delta l_{M}) - l_{max}, &li_{M} + \Delta l_M \geq l_{max}
		\end{cases}
\end{align*}

These sets of equations describe the monitoring stalls for each node in the MFG.
The worst-case monitoring stall cycles can then be determined by maximizing the
value of each $s_M$. This is equivalent to a single objective of maximizing the
sum of the $s_M$,
\begin{align*}
  \max \sum_{M \in \mathcal{M}_{MFG}} s_{M}
\end{align*}
where $\mathcal{M}_{MFG}$ is the set of nodes in the MFG.  Once the worst-case
stalls for each MFG node is found, the worst-case stalls for a CFG node,
$s_{B,max}$, can be computed by simply summing the stalls from the
corresponding MFG nodes. We note that since the monitoring load is always
conservative in representing the FIFO state, no timing anomalies are exhibited
by this analysis. That is, determining the individual worst-case stalls results
in the global worst-case stalls.

\subsection{MILP Formulation}
\label{sec:monitoring_wcet.wcet.milp}

The proposed FIFO model requires solving an optimization problem to obtain the
worst-case stalls, where the input and output monitoring loads, $li_{M}$ and
$lo_{M}$, and the monitoring stalls, $s_{M}$, need to be determined for each
node. Here, we show how the problem can be formulated using MILP. Although the
equations for $lo_{M}$ and $s_{M}$ are non-linear, they are piecewise linear.
Previous work has shown that linear constraints for piecewise linear functions
can be formulated using MILP \cite{sierksma-lp}. In the following constraints,
all variables are assumed to be lower bounded by zero unless otherwise
specified, as is typically assumed for MILP.

First, a set of variables, $lo'$ and $s'$, are created to represent the
unbounded versions of $lo$ and $s$. For readability, the per block subscript
$M$ has been omitted.
\begin{align*}
  s' =& li + \Delta l - l_{max}, \text{ } s' \in (-\infty, \infty)\\
  lo' =& li + \Delta l, \text{ } lo' \in (-\infty, \infty) 
\end{align*}
The following piecewise linear function calculates $s$ from $s'$.
\begin{align*}
  s = f(s') = 
    \begin{cases} 
    0, &s' < 0\\
    s', &s' \geq 0
    \end{cases}
\end{align*}
This function can be described in MILP using the following set of constraints.
\begin{align*}
  a_s\lambda_0 + b_s\lambda_2 =& s'\\
  \lambda_0 + \lambda_1 + \lambda_2 =& 1 \\
  \delta_1 + \lambda_2 \leq& 1 \\
  \delta_2 + \lambda_0 \leq& 1 \\
  \delta_1 + \delta_2 =& 1 \\
  b_s\lambda_2 =& s
\end{align*}
where $a_s$ is chosen to be less than the minimum possible value of $s'$ and
$b_s$ is chosen to be greater than the maximum possible value of $s'$. The
choice of $a_s$ and $b_s$ is arbitrary as long as it meets these requirements.
$\lambda_i$ are continuous variables and $\delta_i$ are binary variables. In
this set of constraints, $s'$ is expressed as a sum of the endpoints of a
segment of the piecewise function. The $\delta_i$ variables ensure that only
the segment corresponding to $s'$ is considered. $\delta_1 = 1$ corresponds to
the $s' < 0$ segment of $f(s')$ and $\delta_2 = 1$ corresponds to the $s' \geq
0$ segment of $f(s')$. The $\lambda_i$ variables represent exactly where $s'$
falls on the domain of that segment. $s$ can be calculated using this
information and the values of the function at the segment endpoints.

Similarly, $lo$ can be bound between $0$ and $l_{max}$ by using the following
set of constraints.
\begin{align*}
  a_l\lambda_3 + l_{max}\lambda_5 + b_l\lambda_6 =& lo'\\
  \lambda_3 + \lambda_4 + \lambda_5 + \lambda_6 =& 1 \\
  2 \delta_3 + \lambda_5 + \lambda_6 \leq& 2 \\
  2 \delta_4 + \lambda_3 + \lambda_6 \leq& 2 \\
  2 \delta_5 + \lambda_3 + \lambda_4 \leq& 2 \\
  \delta_3 + \delta_4 + \delta_5 = 1 \\
  l_{max}\lambda_5 + l_{max}\lambda_6 =& lo
\end{align*}
As before, $a_l$ and $b_l$ are chosen such that $lo' \in (a_l, b_l)$.  Again,
$\lambda_i$ are continuous variables and $\delta_i$  are binary variables.

Finally, for each node, the input monitoring load $li_{M}$ must be determined.
$li_{M}$ depends on the previous nodes, $\mathcal{M_\text{prev}}$. If there is
only one edge into the node, then $li_M$ is simply
\begin{align*}
  li_{M} = lo_{M_{prev}}
\end{align*}
When there is more than one edge into node $M$, one set of constraints is used
to lower bound $li_{M}$ by all $lo_{M_{prev}}$.
\begin{align*}
  li_{M} \geq& lo_{M_{prev}}, \text{ } \forall M_{prev} \in \mathcal{M_\text{prev}}
\end{align*}
Then, another set of constraints upper bounds $li_{M}$ by the maximum $lo_{M_{prev}}$.
\begin{align*}
  li_{M} - b \cdot \delta_{M_{prev}} \leq& lo_{M_{prev}}, \text{ } \forall M_{prev} \in \mathcal{M_\text{prev}} \\
  \sum_{M_{prev} \in \mathcal{M_\text{prev}}}\delta_{M_{prev}} =& |\mathcal{M_\text{prev}}| - 1
\end{align*}
where $b$ is chosen to be greater than $[\max(lo_{M_{prev}}) -\\
\min(lo_{M_{prev}})]$ and $|\mathcal{M_\text{prev}}|$ is the number of nodes
with edges pointing to $M$. $\delta_i$ are binary variables. The use of the
binary variables $\delta_i$ and the second constraint ensure that $li_{M}$ is
only upper bound by one of the $lo_{M_{prev}}$. In order for all constraints to
hold, this must be the maximum $lo_{M_{prev}}$. Together with the lower bound
constraints, these constraints result in $li_{M} = \max(lo_{M_{prev}})$. 


    \end{scriptsize}
    \vspace{-0.1in}
    \caption{Estimated and observed WCET (clock cycles) with and without monitoring.}
    \label{tab:evaluation.wcet}
    \vspace{-0.2in}
  \end{center}
\end{table*}

% Calculated ratios
\begin{table*}[htb]
  \begin{center}
    \begin{tiny}
    % WCET results 

\begin{tabular}{|lll||r|r|r|r|r|r|r||r|r|r|}
\hline

\multicolumn{3}{|c||}{\multirow{2}{*}{\bf Ratio}}  &\multicolumn{7}{c||}{\bf Benchmark}      &\multicolumn{1}{c|}{\multirow{2}{*}{\bf min}}&\multicolumn{1}{c|}{\multirow{2}{*}{\bf max}}&\multicolumn{1}{c|}{\multirow{2}{*}{\bf geomean}} \\ \cline{4-10}
&&&\multicolumn{1}{c|}{\tt cnt}&\multicolumn{1}{c|}{\tt expint}&\multicolumn{1}{c|}{\tt fdct}&\multicolumn{1}{c|}{\tt fibcall}&\multicolumn{1}{c|}{\tt insertsort}&\multicolumn{1}{c|}{\tt matmult}&\multicolumn{1}{c||}{\tt ns}&&& \\ \hline \hline
wcet-none&:&sim-none&1.03&1.52&1.00&1.00&1.00&1.00&1.00&1.00&1.52&1.07 \\ \hline
wcet-umc&:&sim-umc&1.03&1.52&1.18&1.00&1.12&1.09&1.00&1.00&1.52&1.12 \\ \hline
wcet-cfp&:&sim-cfp&1.29&1.71&1.00&1.43&1.13&1.00&1.39&1.00&1.71&1.26 \\ \hline \hline
sequential-umc&:&wcet-umc&1.60&1.03&1.44&1.05&1.19&1.40&1.74&1.03&1.74&1.33 \\ \hline
sequential-cfp&:&wcet-cfp&1.62&1.30&1.09&1.45&1.73&1.73&1.37&1.09&1.73&1.45 \\ \hline \hline
wcet-umc&:&wcet-none&1.00&1.00&1.68&1.00&3.48&1.92&1.00&1.00&3.48&1.41 \\ \hline
wcet-cfp&:&wcet-none&1.45&2.58&1.00&2.23&1.13&1.00&2.29&1.00&2.58&1.55 \\ \hline

\end{tabular}

    \end{tiny}
    \vspace{-0.1in}
    \caption{Ratios comparing results from different experiments.} \label{tab:evaluation.ratios}
    \vspace{-0.3in}
  \end{center}
\end{table*}

\vspace{-0.0in}
\subsection{Experimental Setup}

Our toolflow for the proposed WCET method is shown in Figure~\ref{fig:monitoring_wcet.evaluation.toolflow}. We
first use Chronos \cite{chronos-tool}, an open source WCET tool, to estimate
the WCET for the main task and the monitoring tasks. We also modified Chronos
to produce a MFG of the main task. This MFG and the monitoring task WCET are used to produce an MILP formulation as in
Section \ref{sec:formulation}. This MILP problem is solved
using lp\_solve \cite{lpsolve}, which produces the worst-case monitoring stall cycles for each forwarded
instruction. These monitoring stalls are combined into the ILP formulation that 
is originally generated for the main task to estimate the overall
WCET with parallel run-time monitoring. Although we use Chronos and lp\_solve
for our implementation, these components can be replaced with any WCET
estimation tool and LP solver respectively.

To evaluate the effectiveness of our WCET scheme, we compared its estimate
with a simple WCET bound from sequential monitoring (Section~\ref{sec:formulation:sequential})
as well as simulation results using the SimpleScalar tool \cite{simplescalar}.
In addition to the WCET estimates with monitoring, we also compared the results with
the WCET of the main task without monitoring, using both Chronos and simulations. 
% The baseline WCET allows us to study the overheads of parallel run-time monitoring
% in terms of the worst-case execution time.

%In order to examine
%how conservative the estimate was, we used the SimpleScalar simulator to
%simulate the benchmarks, both with and without monitoring. 

For the experiments, we configured Chronos and SimpleScalar to model simple processing cores
that execute one instruction per cycle for both main and monitoring cores and used an 8-entry FIFO.
This configuration represents typical embedded microcontrollers, and is designed to focus on 
the impact of parallel run-time monitoring by removing complex features such as branch prediction 
and caches.
%The experiments model an 8-entry FIFO that can buffer up to eight
%forwarded instructions. 
In the evaluation, we used seven benchmarks from the M\"alarden WCET benchmark suite \cite{malarden} 
and two monitoring techniques: uninitialized memory checks (UMC) and control flow protection (CFP).
UMC detects a software bug that reads memory without a write as briefly explained in 
Section~\ref{sec:arch}. CFP protects a program's control flow by checking a target address on
each control transfer \cite{arora-runtime05}. In this technique, a compiler determines a set of valid
targets for each branch and jump in the main task.
This information is stored on the monitoring core. 
On a branch or jump, the monitoring core ensures that the target is
contained in the list of valid targets.


\vspace{-0.05in}
\subsection{Results}

Table~\ref{tab:evaluation.wcet} shows the experimental results for each
benchmark under different configurations. The first set of rows show the WCET 
estimate from Chronos ({\tt wcet-none}) and actual run-times from simulations ({\tt sim-none}) without 
monitoring. The remaining rows show the WCET for the UMC and
CFP monitoring extensions. The results are shown for three different approaches:
a bound from sequential monitoring ({\tt sequential}), our approach ({\tt wcet}),
and simulations ({\tt sim}). The numbers indicate the number of clock cycles.
Appendix~\ref{sec:lptime} includes running times for these experiments.

Table~\ref{tab:evaluation.ratios} shows relative comparisons between
different configurations or WCET methods.
The first set of rows compare the WCET estimates from ILP or MILP formulations
with the worst-case simulation cycles for each monitoring setup. 
The results show that the analytical WCET estimates from our proposed scheme
are larger than the observed WCET by 0\% to 52\% for UMC and 0\% to 71\% for CFP, 
depending on the main task. This difference is comparable to the case without
parallel run-time monitoring, where the analytical WCET from Chronos
is larger than simulation results by 0\% to 52\%. 
In fact, for {\tt expint}, the majority of the difference is from the WCET
estimate of the main task rather than the effects of monitoring.
%In fact, for certain benchmarks, such as {\tt expint}, the majority of the difference 
%is due to estimating the
%WCET of the main task rather than the effects of monitoring. 
This result suggests that our WCET approach is not significantly more conservative than
the baseline WCET tool for the main task.

The second set of rows compare the bound from sequential monitoring and the WCET 
from our proposed method. 
For UMC, our approach shows up to a 74\% reduction in WCET estimates over the simple
bound. Similarly, for CFP, our method shows up to a 73\% improvement.
These results demonstrate that modeling the FIFO decoupling between the main and monitoring
tasks is important for obtaining tight WCET estimates of parallel
monitoring. 

Finally, the last two rows in Table~\ref{tab:evaluation.ratios} compare the WCET 
estimates with and without run-time monitoring.
The results show that the increase in WCET varies significantly depending on
benchmark and monitoring technique. Benchmarks with infrequent monitoring
events (forwarded instructions) show minimal overheads while ones with frequent
monitoring can see significant impacts.
Also, the benchmarks with large WCET increases differ between UMC and CFP.
Therefore, when applying parallel run-time monitoring techniques to real-time systems,
a careful WCET analysis for the given tasks and monitoring techniques 
needs to be performed. 

The impact of run-time monitoring on the execution time in our experiments
(up to 3.48x in UMC and 2.58x in CFP) is roughly in line with previous studies
on multi-cores without any hardware support \cite{chen08-lba, nagarajan08-dift}. 
The performance overheads will be much lower for multi-cores with optimizations 
\cite{chen08-lba} or heterogeneous monitors \cite{flexcore-micro10}. 
Our analysis technique does not depend on any specific monitoring core
microarchitecture and is applicable to more optimized architectures.

%Table~\ref{tab:evaluation.ratios} also compares the WCET estimated with each
%monitoring extension versus the WCET estimated without monitoring. For UMC, the
%WCET is increased by up to 3.48x and for CFP the WCET is increased up to 2.58x.
%However, for both extensions, there also exist benchmarks where the WCET is not
%increased. 

%%%%%%%%%%%%%%%%%%%%%%%%%%%%%%%%%%%%%%%%%%%%%%%%%%
% lp_solve time
%%%%%%%%%%%%%%%%%%%%%%%%%%%%%%%%%%%%%%%%%%%%%%%%%%

\subsection{Time to Solve Linear Programming Problem}
\label{sec:lptime}

% lp_solve time
\begin{table*}[htb]
  \begin{center}
    \begin{small}
    % lp_solve run time

\begin{tabular}{|l||r|r|r|r|r|r|r|}
\hline

\multirow{2}{*}{\bf Solver Target}&\multicolumn{7}{c|}{\bf Benchmark}       \\ \cline{2-8}
&\multicolumn{1}{c|}{\tt cnt}&\multicolumn{1}{c|}{\tt expint}&\multicolumn{1}{c|}{\tt fdct}&\multicolumn{1}{c|}{\tt fibcall}&\multicolumn{1}{c|}{\tt insertsort}&\multicolumn{1}{c|}{\tt matmult}&\multicolumn{1}{c|}{\tt ns}\\ \hline \hline
stall-umc&17.789&6.256&21.733&0.043&0.390&161.796&3.655\\ \hline
stall-cfp&3.691&97.93&0.038&0.024&0.025&14.209&1.474\\ \hline \hline
sequential-umc&0.006&0.004&0.004&0.005&0.002&0.004&0.006\\ \hline
sequential-cfp&0.007&0.001&0.003&0.002&0.003&0.006&0.003\\ \hline \hline
wcet-none&0.003&0.003&0.004&0.002&0.002&0.002&0.001\\ \hline
wcet-umc&0.004&0.004&0.003&0.001&0.004&0.005&0.002\\ \hline
wcet-cfp&0.002&0.007&0.005&0.004&0.003&0.005&0.004\\ \hline

\end{tabular}

    \end{small}
    \vspace{-0.1in}
    \caption{Running time of lp\_solve in seconds to determine worst-case stalls
    (stall), sequential bound (sequential), and worst-case execution times (wcet).}
    \label{tab:appendix.runtime}
    \vspace{-0.2in}
  \end{center}
\end{table*}

The most time intensive portion of the WCET analysis is the actual solving of
the linear programming (LP) problem. For our experiments, we used lp\_solve
5.5.2.0 \cite{lpsolve} as our LP solver. These experiments were run on a 2.67
GHz Xeon E5430 quad-core processor with 4 GB of RAM. The running times for
lp\_solve are shown in Table~\ref{tab:appendix.runtime}.  The first set of rows
show the running time for determining the worst-case stalls from the monitoring flow
graph ({\tt stall}). The second set of rows show the lp\_solve running time for
finding the sequential bounds. The final set of rows show the running time for
determining the overall WCET ({\tt wcet}). For {\tt wcet-umc} and {\tt
wcet-cfp}, this is for the ILP problem given the worst-case stalls .

The running times for the {\tt sequential} cases and the {\tt wcet} cases are very
similar. This is because these cases are all solving essentially the same
problem with different numbers. That is, for a given benchmark, these different
cases are all solving a linear programming problem for the same control flow
graph (CFG). As a result, the number of variables and the set of
constraints is the same, though the WCET for each basic block changes depending
on the extension and the estimation method. The {\tt stall} cases have a longer
running time.
This is due to the fact that a MFG has more nodes than its corresponding CFG.
The increased number of nodes also implies more variables and more constraints.


