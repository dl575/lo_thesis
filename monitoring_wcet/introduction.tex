\section{Introduction}
\label{sec:monitoring_wcet.introduction}

In order to apply parallel run-time monitoring to traditional real-time systems, we must
be able to analyze the worst-case execution time (WCET) of running a program
with run-time monitoring. The only existing way to do this is to assume that
monitoring is done sequentially. However, this bound is overly conservative when used to analyze parallel run-time monitoring.
In this chapter, we present a method for estimating the increase in WCET of
programs running on a system with parallel monitoring. We first investigate how
to mathematically model the loosely-coupled relationship between the main 
processing core and parallel monitoring hardware, which are connected
through a FIFO buffer with a fixed number of entries. The resulting model
is non-linear but can be transformed into a mixed integer linear programming (MILP) 
formulation.
The MILP formulation produces the maximum number of cycles for each basic block
that the main core may be stalled due to monitoring.
These monitoring stalls can be incorporated into popular
WCET analysis methods based on implicit path enumeration techniques (IPET)
\cite{li-ipet-dac95} in order to estimate the WCET of tasks with run-time monitoring.

The rest of this chapter is organized as follows.
%Section~\ref{sec:monitoring_wcet.monitoring} gives an overview of the run-time
%monitoring architecture that we assume in this thesis.
Section~\ref{sec:monitoring_wcet.wcet} explains our MILP-based formulation for
estimating WCET and Section~\ref{sec:monitoring_wcet.example} provides an
example MILP formulation. Finally, Section~\ref{sec:monitoring_wcet.evaluation}
shows our evaluation results.


