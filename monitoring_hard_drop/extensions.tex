\section{Monitoring Extensions}
\label{sec:monitoring_hard_drop.extensions}

% Monitoring task classification
\begin{table}
  \begin{center}
    \caption{Monitoring tasks for UMC, RAC, and DIFT.}
    \begin{footnotesize}
    
% Classification of monitoring tasks

\begin{tabular}{|l|l|l|l|}
\hline

{\bf Monitor} & {\bf Event} & {\bf Task} & {\bf Operation} \\ \hline \hline

\multirow{2}{*}{UMC}  
 & Store & Set metadata tag (data initialized) & Update \\ \cline{2-4}
 & Load & Check that metadata tag is set & Check \\ 
\hline\hline

\multirow{2}{*}{RAC}  
 & Call & Push return address & Update \\ \cline{2-4}
 & Return & Pop return address and check & Update \& Check \\ 
\hline\hline

\multirow{4}{*}{DIFT}  
 & ALU & Set the output tag as the OR of the input tags & Update \\ \cline{2-4}
 & Load & Set the register tag as the tag of the loaded data & Update \\ \cline{2-4}
 & Store & Set the memory tag as the register tag & Update \\ \cline{2-4}
 & Indirect jump & Check that tag is not set (data is untainted) & Check \\ \hline

\end{tabular}

    \end{footnotesize}
    \label{tab:monitoring_hard_drop.extensions.tasks}
  \end{center}
\end{table}

% Hardware operations
\begin{table}
  \begin{center}
    \caption{Operation of the MIM for UMC, RAC, and DIFT.}
    \begin{footnotesize}
    % Operation of MIM

\begin{tabular}{|l|l||l|}
\hline

{\bf Monitor} & {\bf Event} & {\bf Invalidation} \\ \hline \hline

\multirow{2}{*}{UMC}  
& Store & Set invalidation flag \\ \cline{2-3}
& Load & Do nothing \\ 
\hline\hline

\multirow{2}{*}{RAC}  
& Call & Invalidate metadata, increment metadata address \\ \cline{2-3}
& Return & Decrement metadata address \\ 
\hline\hline

\multirow{4}{*}{DIFT}  
& ALU & Clear taint in MISP \\ \cline{2-3}
& Load & Clear taint in MISP \\ \cline{2-3}
& Store & Invalidate metadata \\ \cline{2-3}
& Indirect jump & Do nothing \\ \hline

\end{tabular}

    \end{footnotesize}
    \label{tab:monitoring_hard_drop.extensions.mim}
  \end{center}
\end{table}
\begin{table}
  \begin{center}
    \caption{Operation of the MFM for UMC, RAC, and DIFT.}
    \begin{footnotesize}
    % Operation of MFM

\begin{tabular}{|l|l||l|l|}
\hline

{\bf Monitor} & {\bf Event} & {\bf Filter Condition} & {\bf Filter Operation} \\ \hline \hline

\multirow{2}{*}{UMC}  
& Store & Never & - \\ \cline{2-4}
& Load & Invalid metadata & Do nothing \\ 
\hline\hline

\multirow{2}{*}{RAC}  
& Call & Never & - \\ \cline{2-4}
& Return & Invalid metadata & Decrement stack pointer \\ 
\hline\hline

\multirow{4}{*}{DIFT}  
& ALU & Always & Perform propagation \\ \cline{2-4}
& Load & Invalid metadata & Clear taint in MISP \\ \cline{2-4}
& Store & Never & - \\ \cline{2-4}
& Indirect jump & Invalid metadata & Do nothing \\ \hline

\end{tabular}

    \end{footnotesize}
    \label{tab:monitoring_hard_drop.extensions.mfm}
  \end{center}
\end{table}

In this section, we detail how our architecture works for the three different
monitoring schemes we evaluated: uninitialized memory check (UMC), return
address check (RAC), and dynamic information flow tracking (DIFT).
Table~\ref{tab:monitoring_hard_drop.extensions.tasks} shows a summary of the
monitoring tasks for these monitoring schemes.
Table~\ref{tab:monitoring_hard_drop.extensions.mim} shows how the MIM operates
and Table~\ref{tab:monitoring_hard_drop.extensions.mfm} shows how the MFM
operates for each monitoring scheme. 

\subsection{Uninitialized Memory Check}
\label{sec:monitoring_hard_drop.extensions.umc}

Uninitialized memory check (UMC) is a monitoring scheme that checks that memory
is written to before it is read from.
Table~\ref{tab:monitoring_hard_drop.extensions.tasks} summarizes the monitoring
tasks for UMC.

Tables~\ref{tab:monitoring_hard_drop.extensions.mim} and
\ref{tab:monitoring_hard_drop.extensions.mfm} shows the operations of
the MIM and MFM. The monitoring task performed on a load instruction is a check
operation.  Thus, it can simply be skipped without causing any false positives.
However, the monitoring task on a store updates metadata. On a dropped store
event, the MIM calculates the metadata address and sets an invalidation flag
corresponding to this address in the MIT to indicate that the metadata for this
word is invalid.

In terms of using the metadata filtering module, on a load, if the metadata is
invalid, then performing the check is useless. Thus, if the filtering module
detects that the metadata invalidation flag corresponding to the memory access
address is invalid, then the monitoring event is dropped and no operation is
performed.  For a store event, the source operand of the operation is a
register value which is always valid. Thus, the MFM never filters a store
event. Instead, if the store event is not dropped, the monitoring task will set the
metadata bit and can clear the corresponding invalidation flag.

\subsection{Return Address Check}
\label{sec:monitoring_hard_drop.extensions.lrc}

Several security attacks, such as return-oriented programming (ROP)
\cite{rop-ccs07}, attempt to manipulate the return address of a
function so that the control flow of a program is diverted. One method to
prevent these types of attacks is to use monitoring to save the return address
on a call to a function and to check on the return that the correct return
address is used. We refer to this type of monitoring scheme as a return address
check (RAC). On a call instruction, RAC pushes the correct return address for
the function onto a stack data structure. On a return instruction, RAC pops an
address off the stack and checks that this saved address matches the address the
main task is returning to (Table~\ref{tab:monitoring_hard_drop.extensions.tasks}).

In the case of UMC, the metadata that we cared about was a function of
information from the monitoring event. For RAC, the metadata entry that we care
about is based on the current stack pointer for the return address stack that
the monitor maintains.  The MIM maintains the previous metadata address used in
a register and we can use this register to maintain the stack pointer between
the monitoring task and the MIM. The monitoring task updates this register as
necessary to ensure that the MIM has the correct metadata stack pointer. On a
call instruction that needs to be dropped, the MIM marks the metadata at the
current stack pointer (i.e., the previous metadata address) as invalid and
increments the stack pointer. On a return, the MIM decrements the stack pointer
and simply skips the check operation
(Table~\ref{tab:monitoring_hard_drop.extensions.mim}).

On a call instruction, the source of our monitoring operation is the return
address from the monitoring event. This is always valid, so call instructions
are never processed by the MFM. On a return instruction, if the stored return
address is marked as invalid, then the check can be skipped. Thus, if the MFM
detects that the stored return address is invalid, then the stack pointer will
be decremented but no check will occur, similar to the invalidation case.

\subsection{Dynamic Information Flow Tracking}
\label{sec:monitoring_hard_drop.extensions.dift}

Dynamic information flow tracking (DIFT) \cite{dift-asplos04} is a security
monitoring scheme that seeks to detect when information from untrusted sources
is used to affect the control flow.  In its simplest form, DIFT keeps a 1-bit
metadata tag for each memory location and each register, indicating tainted or
untainted. Multi-bit versions of DIFT have been proposed to track more detailed
semantic information about data. When data is read from an untrusted source, it
is marked as tainted. As this data is used for other operations, the results of
these operations are also marked as tainted. On an indirect jump, the taint bit
of the register used is checked. If the register is found to be tainted, then
an error is detected. These monitoring task operations are summarized in
Table~\ref{tab:monitoring_hard_drop.extensions.tasks}.

In the case of DIFT, register metadata is used often, so we will use the MISP
to store the register metadata invalidation information in order to avoid aliasing.  Furthermore,
in this discussion, we are considering a 1-bit taint, so the MISP can actually
be used to store the exact
taint value, rather than an invalidation flag. This shows some of the
flexibility in how these hardware structures can be used for specific
monitoring schemes. For more complicated multi-bit DIFT schemes, the MISP would
be used to store invalidation flags as before.  For ALU and load instructions
which have a register as a destination, the MIM will clear the register's taint
in the MISP on a dropped monitoring event. This is safe because an error is
only raised for a tainted data value. On a store instruction, which writes to
memory as a destination, the MIM invalidates the associated memory address in
the MIT, similar to how UMC operates. Finally, on an indirect jump, only a
check operation is performed. Thus, this can be dropped without any extra
invalidation. 

The MFM operates similar to the UMC case for some of these monitoring events
(Table~\ref{tab:monitoring_hard_drop.extensions.mfm}).  For an indirect jump,
if the metadata is found to be invalid, then the event is filtered and no check
is performed. For a load instruction, if the loaded data has its metadata
marked as invalid, then the MFM will clear the taint bit for the target
register, as in the invalidation case. For a store instruction, the source is
from a register, which always has a valid taint bit in the MISP, so these are
never filtered out. The most interesting case is for ALU instructions. The MFM
is able to read the taint bits from the MISP and then signal to the MIM to
write to the MISP, so we can actually perform the entire monitoring task in
hardware. The MFM reads the taint bits from the MISP for both source registers.
These are fed into the filter lookup table (see
Figure~\ref{fig:monitoring_hard_drop.hwdrop.mfm}) which is configured to signal
the MIM to set or clear the taint bit in the MISP for the destination register
depending on the read taint bits.

