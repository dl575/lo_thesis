\section{Introduction}
\label{sec:monitoring_hard_drop.introduction}

% Run-time monitoring has high WCET.
In the last chapter, we showed how the WCET of run-time monitoring could be
estimated. By accounting for this WCET, this allows run-time monitoring to be
implemented in traditional real-time frameworks while giving strong guarantees
that deadlines will be met. However, the increase in WCET introduced by
run-time monitoring can be quite high, requiring a large amount of slack in a
system's schedule in order to support it. For example,
Section~\ref{sec:monitoring_hard_drop.evaluation} showed that even an
aggressive implementation of UMC showed increased WCET of 3.5x compared to
without monitoring. A more realistic implementation that includes deciding
which tasks to run in software can show increases of WCET of up to 12x compared
to the baseline.  Thus, applying monitoring to real-time systems requires a
large increase in the time allocated to tasks.  Currently, if a real-time
system cannot support this extra utilization, then monitoring cannot be applied
to the system.  

% Our solution is based on dynamic slack.
In this chapter, we describe a system that greatly decreases the amount of time
that must be statically allocated to tasks in the worst case (i.e., WCET) in
order to enable monitoring. Specifically, this system exploits dynamic slack in
order to opportunistically perform as much monitoring as possible within the
time constraints of the system. This is based on three key observations:
\begin{enumerate}
  \item Tasks often run faster than their worst-case time.
  \item The performance impact of monitoring is rarely equal to the worst-case impact.
  \item Performing partial monitoring can still provide some degree of protection.
\end{enumerate}
Since tasks typically run faster than their conservatively estimated WCET,
there exists dynamic slack at run-time that can be used for monitoring.
Similarly, the estimation of the WCET of monitoring is conservative. In
actuality, the overheads of monitoring are much lower, as shown by the
average-case performance impact. Finally, although performing all monitoring
operations in full is preferred, performing only a portion of the monitoring
still gives increased protection over a system with no monitoring applied.
By shifting the decision of whether to enable or disable monitoring to
run time, we are able to trade off monitoring coverage in order to reduce the
WCET impact of monitoring. The system we present decides whether or not to
perform monitoring based on whether there is enough dynamic slack to account
for the worst-case performance impact of the monitoring.

% Key issue is false positives
A main challenge in skipping monitoring operations is ensuring that the monitor
can still run in a useful manner. Although we trade off the coverage provided
by monitoring in order to meet timing requirements, we must also  guarantee
that no false positives occur in order to prevent false alarms.
We prevent false positives through the use of a dropping task that invalidates
metadata when monitoring is skipped.
With hardware optimizations, this invalidation can be handled with no impact on
the task's WCET. In addition, the hardware architecture we present skips
monitoring on invalidated metadata, saving dynamic slack to be used for useful
monitoring on valid metadata.

The rest of this chapter is organized as follows. In
Section~\ref{sec:monitoring_hard_drop.drop}, we discuss how to decide when to
skip monitoring and how to handle this in a safe manner. Hardware optimizations
for handling this dropping operation are presented in
Section~\ref{sec:monitoring_hard_drop.hwdrop}. We present our evaluation
methodology and experimental results in
Section~\ref{sec:monitoring_hard_drop.evaluation}. 

