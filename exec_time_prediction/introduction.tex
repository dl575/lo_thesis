\section{Introduction}
\label{sec:exec_time_prediction.introduction}

In the previous chapters, we have explored improving the security and
reliability of real-time systems. We have looked at the challenges and design
choices involved with applying run-time monitoring to real-time systems.
However, information about real-time requirements can also provide an
opportunity with which to better optimize designs. In this chapter, we discuss
how soft real-time constraints can be used to inform the control of
dynamic voltage and frequency scaling (DVFS) for improved energy-efficiency.

Many modern applications on mobile and desktop systems are interactive.
That is, users will provide inputs and expect a response in a timely manner.
For example, games must read in user input, update game state, and display the
result within a small time window in order for the game to feel responsive.
Previous studies on human-computer interaction have shown that latencies of 100
milliseconds are required in order to maintain a good experience
\cite{endo-osdi96, card-chi91, miller-afips68} while variations in response
time faster than 50 milliseconds are imperceptible \cite{lindgaard-bit06,
eqos-hpca15}. These tasks are effectively soft real-time tasks which have a
user response-time deadline. The task must finish by the deadline for good user
experience, but finishing faster does not necessarily improve the user
experience due to the limits of human perception.

As finishing these tasks faster is not beneficial, we can use power-performance
trade-off techniques, such as dynamic voltage and frequency scaling (DVFS), in
order to reduce energy usage while maintaining the same utility to the user. By
reducing the voltage and frequency, we can run the task slower and with less
energy usage. As long as the task still completes by the deadline, then the
experience for the user remains the same. 

Existing Linux power governors \cite{linux_governors} do not take into account
these response-time requirements when adjusting DVFS levels. More recent work
has looked at using DVFS to minimize energy in the presence of deadlines.
However, these approaches are typically reactive, using past histories of task
execution time as an estimate of future execution times \cite{gu-dac08,
choi-iccad02, pegasus-isca14, nachiappan-hpca15}.  However, the execution time
of a task can vary greatly depending on its input.  Reactive controllers cannot
respond fast enough to input-dependent execution time variations.  Instead,
proactive or predictive control is needed in order to adjust the operating
point of a task before it runs, depending on the task inputs. This has been
explored for specific applications \cite{gu-rtas08, zhu-hpca13, eqos-hpca15,
adrenaline-hpca15}, but these approaches were developed using detailed analysis
of the applications of interest. As a result, they are not easily generalizable
to other domains.

In this chapter, we present an automated and general framework for creating
prediction-based DVFS controllers. Given an application, we show how to create
a controller to predict the appropriate DVFS level for each task execution,
depending on its input and program state values, in order to satisfy
response-time requirements. Our approach is based on recognizing that the
majority of execution time variation can be explained by differences in control
flow.  Given a task, we use program slicing to create a minimal code fragment
that will calculate control-flow features based on input values and current
program state. We train a model to predict the execution time for a task given
these features. At run-time, we are able to quickly run this code fragment and
predict the execution time of a task. With this information, we can
appropriately select a DVFS level in order to minimize energy while satisfying
response-time requirements.  
% Our main contributions include:
% \begin{enumerate}
%   \item An automated approach for generating control flow features.
%   \item A method for training a model to map features to execution times in
%   such a way as to be conservative and avoid deadline misses.
%   \item Application of these predictions to DVFS control in order to minimize
%   energy in the presence of deadlines.
% \end{enumerate}

% We implemented a prototype of this method on an ODROID-XU3 development board.
% We tested eight different applications, including games, speech recognition,
% video decoding, and a web browser. Our results show energy savings of 56\% over
% running at maximum frequency with almost no deadline misses. This is 27\%
% increased energy savings compared to the default Linux interactive governor
% which shows 2\% deadline misses.  Compared to a PID-based governor, our
% approach sees only a 1\% improvement in energy savings but the PID-based
% controller shows 13\% deadline misses on average.

The rest of the chapter is organized as follows.
Section~\ref{sec:exec_time_prediction.applications} discusses general
characteristics of interactive applications and the challenges of designing a
DVFS controller to take advantage of execution time variation.
Section~\ref{sec:exec_time_prediction.prediction} discusses our method of
predicting execution time and using this to inform DVFS control.
Section~\ref{sec:exec_time_prediction.system} discusses the overall framework
and system-level issues. Finally, Section~\ref{sec:exec_time_prediction.evaluation}
shows our evaluation results. 
