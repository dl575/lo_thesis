The last decade has seen an increased ubiquity of computers with the widespread
adoption of smartphones and tablets and the continued spread of embedded
cyber-physical systems. With this integration into our environments, it has
become important to consider the real-world interactions of these computers
rather than simply treating them as abstract computing machines. For example,
cyber-physical systems typically have real-time constraints in order to ensure
safe and correct physical interactions. Similarly, even mobile and desktop
systems, which are not traditionally considered real-time systems, are
inherently time-sensitive because of their interaction with users.  Traditional
techniques proposed for improving hardware architectures are not designed with
these real-time constraints in mind. In this thesis, we explore some of the
challenges and opportunities in hardware design for real-time systems.
Specifically, we study recent techniques for improving security, reliability,
and energy-efficiency of computer systems. We analyze and adapt recently
proposed run-time monitoring techniques for improving security and reliability
to real-time systems. In addition, we show how to take advantage of dynamic
voltage and frequency scaling in the presence of timing requirements for
improved energy-efficiency.

