The last decade has seen an increased ubiquity of computers with the widespread
adoption of smartphones and tablets and the continued spread of embedded
cyber-physical systems. With this integration into our environments, it has
become important to consider the real-world interactions of these computers
rather than simply treating them as abstract computing machines. For example,
cyber-physical systems typically have real-time constraints in order to ensure
safe and correct physical interactions. Similarly, even mobile and desktop
systems, which are not traditionally considered real-time systems, are
inherently time-sensitive because of their interaction with users.  Traditional
techniques proposed for improving hardware architectures are not designed with
these real-time constraints in mind. In this thesis, we explore some of the
challenges and opportunities in hardware design for real-time systems.
Specifically, we study recent techniques for improving security, reliability,
and energy-efficiency of computer systems. 

Run-time monitoring has been shown to be a promising technique for improving
system security and reliability. Applying run-time monitoring to real-time
systems introduces challenges due to the performance impact of monitoring.
To address this, we first developed a technique for estimating the worst-case
execution time impact of run-time monitoring, enabling it to be applied within
existing real-time system design methodologies. Next, we developed hardware
architectures that selectively enable and disable monitoring in order to meet
real-time deadlines while still allowing a portion of the monitoring to be
performed. We show architectures for hard and soft real-time systems, with
different design decisions and trade-offs for each.

Dynamic voltage and frequency scaling (DVFS) is commonly found in modern
processors as a way to dynamically trade-off power and performance.
In the presence of real-time deadlines, this presents an opportunity
for improved energy-efficiency by slowing down jobs to reduce energy while
still meeting deadline requirements. We developed a method for creating DVFS
controllers that are able to predict the appropriate frequency level for tasks
before they execute and show improved energy savings and reduced deadline
misses compared to current state-of-the-art DVFS controllers.

