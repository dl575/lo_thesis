The last decade has seen an increased ubiquity of computers with the widespread
adoption of smartphones and tablets and the continued spread of embedded
cyber-physical systems. With this integration into our environments, it has
become important to consider the real-world interactions of these computers
rather than simply treating them as abstract computing machines. For example,
cyber-physical systems typically have real-time constraints in order to ensure
safe and correct physical interactions. Similarly, even mobile and desktop
systems, which are not traditionally considered real-time systems, are
inherently time-sensitive because of their interaction with users.
Traditionally, techniques proposed for improving
hardware architectures are only evaluated for their average-case
performance and are not designed to take into account the possibility of timing
constraints. In my thesis, I propose to explore the development of hardware
architectures that are aware of real-time requirements. Specifically, I plan to
investigate the challenges and opportunities in adapting hardware techniques
for improving system security, reliability, and energy-efficiency to real-time
contexts.


