\section{Partial Run-Time Monitoring}
\label{sec:monitoring_dift_drop.monitoring}

% Previous work overheads
\begin{table}
  \begin{center}
    \begin{footnotesize}
    %\begin{tabular}{|l|p{2in}|r|r|}
\begin{tabular}{|l|l|r|}

\hline
\multirow{2}{*}{\bf Name} & {\bf Type, monitoring scheme,} & {\bf Slowdown} \\ 
& {\bf and flexibility} & {\bf (average/worst)} \\ \hline\hline

% DIFT \cite{dift-asplos04} & Custom HW & DIFT only & 1.1\% / 23\% \\ \hline
% FlexiTaint \cite{flexitaint-hpca08} & Custom HW & DIFT w/ programmable policies & 1.1\%-3.7\% / 8.7\% \\ \hline
% Hardbound \cite{hardbound-asplos08} & Custom HW & Array bounds checks only & 5\%-9\% / 22\% \\ \hline
% Harmoni \cite{harmoni-dsn12} & Custom HW & Tag-based monitors & 1\%-10\% / 20\% \\ \hline\hline

\multirow{2}{*}{DIFT \cite{dift-asplos04}} & Custom HW & \multirow{2}{*}{1.01x / 1.23x} \\ 
& DIFT only & \\ \hline
\multirow{2}{*}{FlexiTaint \cite{flexitaint-hpca08}} & Custom HW & \multirow{2}{*}{1.01x-1.04x / 1.09x} \\ 
& DIFT w/ programmable policies & \\ \hline
\multirow{2}{*}{Hardbound \cite{hardbound-asplos08}} & Custom HW & \multirow{2}{*}{1.05x-1.09x / 1.22x} \\
& Array bounds checks only & \\ \hline
\multirow{2}{*}{Harmoni \cite{harmoni-dsn12}} & Custom HW & \multirow{2}{*}{1.01x-1.10x / 1.20x} \\
& Tag-based monitors & \\ \hline\hline

\multirow{2}{*}{FlexCore \cite{flexcore-micro10}} & Dedicated FPGA & \multirow{2}{*}{1.05x-1.44x / 1.84x} \\ 
& Instruction-trace monitoring & \\ \hline
\multirow{3}{*}{FADE \cite{fade-hpca14}} & Core+Custom HW & \multirow{3}{*}{1.2x-1.8x / 3.3x} \\ 
& Instruction-trace monitoring  & \\ 
& (effective when HW filters work) & \\ \hline
\multirow{3}{*}{LBA-accelerated \cite{lba-isca08}} & Multi-core+Custom HW & \multirow{3}{*}{1.02x-3.27x / 5x} \\ 
& Instruction-trace monitoring  & \\
& (effective when accelerators work) & \\ \hline
\multirow{2}{*}{LBA \cite{lba-asid06}} & Multi-core+Custom HW & \multirow{2}{*}{3.23x-7.80x / 11x} \\ 
& Instruction-trace monitoring & \\ \hline\hline

\multirow{2}{*}{Multi-core DIFT \cite{nagarajan-interact08}} & SW (multithreaded) & \multirow{2}{*}{1.48x / 2.2x} \\ 
& DIFT (compiled for each application) & \\ \hline
\multirow{2}{*}{LIFT \cite{lift-micro06}} & SW (DBI) & \multirow{2}{*}{3.6x / 7.9x} \\ 
& DIFT (fully flexible) & \\ \hline
\multirow{2}{*}{Purify \cite{purify-usenix92}} & SW (DBI) & \multirow{2}{*}{2.3x / 5.5x} \\ 
& Memory leak checks (fully flexible) & \\ \hline
\multirow{2}{*}{TaintCheck \cite{taintcheck-ndsss05}} & SW (DBI) & \multirow{2}{*}{10x / 27x} \\ 
& DIFT (fully flexible) & \\ \hline

\end{tabular}

    \end{footnotesize}
    \caption{Trade-off between performance overhead and flexibility/complexity
    of run-time monitoring systems.}
    \label{tab:monitoring_dift_drop.monitoring.previous_overheads}
  \end{center}
\end{table}

\subsection{Overhead of Run-Time Monitoring}

There have been a number of proposals for run-time monitoring systems exploring
various design points.
Table~\ref{tab:monitoring_dift_drop.monitoring.previous_overheads} summarizes
some of the representative designs and their reported performance overheads.
The previous studies show that there exist trade-offs between
efficiency, flexibility, and hardware costs.  For example, a run-time
monitoring scheme can often be realized with fairly low performance overhead
(less than 20\%) if implemented with custom hardware that is designed only for
one monitor or a narrow set of monitors. However, the custom hardware monitors
cannot be modified or updated, and require dedicated silicon area.  On the
other hand, flexible systems that support a wide range of monitors lead to
noticeable performance overhead, often too high for wide deployment in
practice.  Software-only implementations \cite{nagarajan-interact08,
lift-micro06, purify-usenix92, taintcheck-ndsss05} or multi-core monitors with
minimal hardware changes \cite{lba-asid06} are reported to have severalfold
slowdowns.  On-chip FPGA monitors \cite{flexcore-micro10} and cores with
monitoring accelerators \cite{lba-isca08, fade-hpca14} can reduce overhead
significantly, but still show slowdowns of several tens of percents in some
cases.  In today's monitoring systems, the overhead is also unpredictable
because it depends heavily on the characteristics of applications and
monitoring operations.  In summary, users currently need to either pay
noticeable overhead or the cost of custom hardware in order to use fine-grained
run-time monitoring in deployed systems.

\subsection{Partial Monitoring for Adjustable Overhead}

Our goal is to develop a general framework that enables monitoring
overhead to be dynamically adjusted by dropping a portion of monitoring
operations if necessary. In essence, this framework adds a new dimension to the
monitor design space by allowing coverage or accuracy to be traded off for
lower overhead. For example, the capability to adjust overhead allows users to
use monitoring with partial coverage in order to reduce performance or energy
overheads. Alternatively, partial monitoring allows designers to use less
expensive hardware for a given performance overhead budget.

In this framework, a user specifies how much monitoring should be done in the
form of a target overhead budget, a target coverage, a percentage of monitoring
operations to be performed, etc.  Then, the framework dynamically drops a
portion of monitoring operations to match the target. In particular, we
focus on matching a performance overhead target while maximizing the
coverage. Given that the overhead of custom hardware monitors can already be
quite low, the focus is on enabling the trade-off in {\em general-purpose}
monitoring systems that support a wide range of monitors.  We also consider the
target overhead as a soft constraint and do not aim to provide a strict
worst-case guarantee.

\subsection{Applications and Metrics}

While it is ideal if run-time monitoring can be performed in full, we believe
that the capability to trade off coverage/accuracy for lower
performance/hardware overhead will be useful in many application scenarios
where full monitoring is not a viable option.  Here, we briefly discuss example
applications of partial monitoring and the metrics that are important in each
case.

{\bf Cooperative testing, debugging, and protection:} 
Recent studies have shown that software testing, debugging, or attack detection
may be done in a cooperative fashion across a large number of systems
\cite{liblit-pldi05, chilimbi-asplos04, greathouse-cgo11, testudo-micro08}. In
this case, each system is only willing to pay very low overhead (e.g., a few
percent) and only performs a small subset of checks.  High coverage is achieved
by having different systems check different parts of a program.  The partial
monitoring framework allows high-overhead monitoring to be used in a
cooperative fashion.  The main metric that represents the effectiveness of
monitoring in this case is the coverage (the percentage of checks that were
performed) over multiple runs.

{\bf Monitoring of soft real-time systems:}
Soft real-time systems or interactive systems need to meet deadlines or
response-time requirements. Unfortunately, the overhead of run-time monitoring
is often unpredictable and varies significantly depending on the application
and monitor characteristics. The partial monitoring framework allows monitoring
to be performed while providing a level of guarantee on its impact on the
execution time.  In this case, it is important that the system can closely
match the desired overhead target while maximizing the effectiveness of
monitoring.

{\bf Partial protection for low overhead:}
Even without real-time constraints, systems may have tight budgets for the
performance, energy, or hardware overheads that they can tolerate for run-time
monitoring. In such cases, full monitoring cannot be enabled unless its
overhead is low enough. Adjustable monitoring allows partial protection on such
systems. For example, array bounds may be checked for a subset of memory
accesses. For dynamic information flow tracking (DIFT), a subset of information
flows may be tracked for partial attack detection. In this scenario, the
effectiveness of partial monitoring can be measured as the percentage of
run-time checks that are performed on each program run, which reflects how
likely it is for a bug or an attack to be detected for a system. 

{\bf Profiling:} 
The run-time monitoring system can be used to implement various profiling tools
to collect statistics on program behavior for performance optimizations as well
as security protection. For example, a recent study showed that an instruction
mix can be used to identify malware from normal programs \cite{demme-isca13,
tang-raid14}.  In such profiling tools, the partial monitoring framework can be
used to obtain statistical samples rather than complete counts of all program
events, essentially trading off accuracy for lower overhead.

\subsection{Design Challenges}

While conceptually simple, designing a general framework to dynamically adjust
monitoring overhead introduces new challenges that need to be addressed.  The
following summarizes the main design goals and associated design challenges.

\begin{enumerate}
  \item \textbf{General:} Since we mainly target flexible run-time monitoring
  systems, which often have high overhead, the framework also needs to be
  general enough to be applicable to a wide range of monitoring schemes.

  \item \textbf{No false positive:} The framework needs to ensure that dropping
  a portion of monitoring operations does not lead to a false positive. We
  found that a dataflow engine that tracks invalid metadata can serve as a
  general solution to this problem
  (Section~\ref{sec:monitoring_dift_drop.dropping}).

  \item \textbf{Intelligent dropping:} The framework needs to match the
  overhead budget while maximizing the effectiveness of monitoring. To this
  end, partial monitoring needs to carefully choose which operations to drop
  and when. We address this problem by studying different dropping policies
  (Section~\ref{sec:monitoring_dift_drop.policies}) and their trade-offs.
\end{enumerate}

