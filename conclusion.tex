\chapter{Conclusion}
\label{chap:conclusion}

% Summary
Traditionally, real-time requirements have been handled at the software level,
with the hardware being agnostic to these timing requirements. However,
hardware operation can affect timing in ways that are not always exposed to
software. In addition, opportunities exist for better designs by exposing these
timing requirements to hardware. In this thesis, we have addressed some of the
challenges and explored some of the opportunities for the hardware design of
real-time systems.

% Security
We have explored how to apply run-time monitoring techniques for improving
security and reliability to real-time systems. We have developed a method
to estimate the worst-case execution time (WCET) of run-time monitoring,
allowing it to be accounted for in real-time scheduling and analysis frameworks
(Chapter~\ref{chap:monitoring_wcet}). Our estimated bounds for run-time
monitoring are within 71\% of observed worst-case performance, similar to the
baseline tools which show up to 52\% differences between simulations and WCET
estimates. We have also developed architectures for applying run-time
monitoring to existing systems with hard real-time deadlines
(Chapter~\ref{chap:monitoring_hard_drop}) or soft performance constraints
(Chapter~\ref{chap:monitoring_dift_drop}). For hard real-time systems, we show
that an average of 15-66\% of monitoring checks, depending on the monitoring
scheme, can be performed with no impact on WCET. Similarly, for
performance-constrained systems, we show average monitoring coverage of 14-85\%
with significantly reduced performance impact compared to performing full
monitoring.

% Energy-efficiency
We have also explored some of the opportunities for reducing the energy usage
of real-time systems. We have shown how soft real-time requirements can be used
to inform the dynamic voltage and frequency scaling of hardware. By developing
a prediction-based DVFS controller, we showed how the appropriate frequency
could be selected for each job in order to minimize energy while meeting its
deadline requirement.  Our experiments showed 56\% average energy savings with
our DVFS controller.

% Closing thoughts
As computing systems become more widespread and more deeply embedded in our
daily lives, their real-time interactions will become increasingly important.
In this thesis, we have explored the use of modern hardware techniques for
improving system security, reliability, and energy-efficiency in the context of
real-time systems. New hardware techniques for improving computer systems are
continually being proposed, especially as we face new challenges with
energy-efficiency and the slowing of Moore's Law. In addition, computing
systems continue to become more deeply in our daily lives. As a result,
designing hardware with real-time requirements will continue to be an avenue of
future research.

