\chapter{Conclusion}
\label{chap:conclusion}

\section{Summary}

% Summary
Traditionally, real-time requirements have been handled at the software level,
with the hardware being agnostic to these timing requirements. However,
hardware operation can affect timing in ways that are not always exposed to
software. In addition, opportunities exist for better designs by exposing these
timing requirements to hardware. In this thesis, we have addressed some of the
challenges and explored some of the opportunities for the hardware design of
real-time systems.

% Security
We have explored how to apply run-time monitoring techniques for improving
security and reliability to real-time systems. We have developed a method
to estimate the worst-case execution time (WCET) of run-time monitoring,
allowing it to be accounted for in real-time scheduling and analysis frameworks
(Chapter~\ref{chap:monitoring_wcet}). Our estimated bounds for run-time
monitoring are within 71\% of observed worst-case performance, similar to the
baseline tools which show up to 52\% differences between simulations and WCET
estimates. We have also developed architectures for applying run-time
monitoring to existing systems with hard real-time deadlines
(Chapter~\ref{chap:monitoring_hard_drop}) or soft performance constraints
(Chapter~\ref{chap:monitoring_dift_drop}). For hard real-time systems, we show
that an average of 15-66\% of monitoring checks, depending on the monitoring
scheme, can be performed with no impact on WCET. Similarly, for
performance-constrained systems, we show average monitoring coverage of 14-85\%
with significantly reduced performance impact compared to performing full
monitoring.

% Energy-efficiency
We have also explored some of the opportunities for reducing the energy usage
of real-time systems. We have shown how soft real-time requirements can be used
to inform the dynamic voltage and frequency scaling of hardware. By developing
a prediction-based DVFS controller, we showed how the appropriate frequency
could be selected for each job in order to minimize energy while meeting its
deadline requirement.  Our experiments showed 56\% average energy savings with
our DVFS controller.

% Closing thoughts
As computing systems become more widespread and more deeply embedded in our
daily lives, their real-time interactions will become increasingly important.
In this thesis, we have explored the use of modern hardware techniques for
improving system security, reliability, and energy-efficiency in the context of
real-time systems. New hardware techniques for improving computer systems are
continually being proposed, especially as we face new challenges with
energy-efficiency and the slowing of Moore's Law. In addition, computing
systems continue to become more deeply in our daily lives. As a result,
designing hardware with real-time requirements will continue to be an avenue of
future research.

\section{Secure and Reliable Real-Time Systems}

In this thesis, we have presented how to apply hardware run-time monitoring to
real-time systems. However, these type of monitoring features have not yet been
implemented in general-purpose commercial processors, much less any that target
real-time systems. Thus, it may be some time before this research becomes
directly applicable. 

However, current trends could cause hardware run-time monitoring to be realized
sooner rather than later. Although Moore's Law has continued increaser
transistor count on chips, it is becoming more difficult to find ways to use
these transistors to improve performance. Additionally, the last few years have
seen numerous high-profile and high-cost security attacks occur. Major
corporations have been attacked and had user information leaked.  The continued
increase in transistor count and the rising importance of computer security
could indicate that we may begin to see hardware security features, such as
run-time monitoring, in commercial processors in the near future. Being able to
apply these techniques to real-time systems will be especially beneficial as
real-time systems are also often safety-critical due to cyber-physical
interactions. These systems have not been immune to attack either as recent
reports of attacks in automobiles and airplanes have surfaced. 

A quicker application of run-time monitoring for real-time systems could in
soft-core processors implemented on FPGAs. This does not require as long a
design cycle as an ASIC processor. In addition, although run-time monitoring
would require resources in an FPGA, it would not require justifying the cost of
silicon for an ASIC design, especially as it may be niche applications which
would seek these added security and reliability benefits at first.

One interesting future direction to explore would be to see if there are
possible gains from combining static analysis of applications with run-time
monitoring. Run-time monitoring can catch certain errors that would not be
caught be static analysis due to incomplete information that is only available
at run-time (e.g., input-dependent data values). However, there are a number of
properties that can be checked statically. If this information of what is
already verified can be passed to the run-time monitoring system, then the
monitoring that must be done at run-time can be reduced and the overheads of
monitoring can be reduced.

\section{Energy-Efficient Real-Time Systems}

In this thesis, we have also shown how response-time requirements can be used
with prediction-based DVFS control in order to reduce energy usage without
impacting user experience. This was an interesting interaction of real-time
constraints with system design. Typically, real-time constraints introduce new
requirements that constrict and degrade the design. Instead, in this case, the
presence of real-time requirements offers an \emph{opportunity} to produce a
better system (i.e., lower energy usage).

Our work presented in this thesis is a first step in developing general and
automated prediction-based resource control. Our results on a real system show
significant savings and indicate that this may be a promising direction of
future research. Similar approaches could be applied to other domains and
components such as GPUs, FPGAs, hardware accelerators, multi-threaded
applications, and/or heterogeneous datacenters. These will require expanding on
the prediction methods presented in this thesis. We have shown one example of
this with with our study of applying the technique to selecting core type for a
system with heterogeneous big and little cores.

It seems promising that this approach can be applied to commercial applications
as we have demonstrated the technique working on applications running on a real
system. The main weakness likely lies in the quality of the tools written. We
have written a set of tools to instrument code for features, perform program
slicing, and generate the resulting predictor. These tools may run into issues
with more complex code bases. However, these tool requirements are not
particularly new. For example, commercial tools exist for performing program
slicing, likely the most complex portion of the toolflow. Thus, the challenges
in creating a more robust toolset are not insurmountable. 

The workloads we tested worked well with using control flow features. However,
it is likely that irregular workloads exist which do not show execution time
that correlates well with control flow features. This will require further
study into what additional features may be useful for prediction, such as
information about memory patterns. In addition, although we have avoided
needing detailed programmer knowledge, it is likely that some amount of hints
from the programmer will be useful in informing the prediction.

\section{Hardware Designs for Real-Time Systems}

The increased integration of computers in our daily lives, especially in the
form of cyber-physical systems, is increasingly creating context where
real-time is an issue for computing. In addition, as hardware becomes
increasingly complex, it is harder to manage the timing of tasks purely from
the software layer. For example, hardware-based run-time monitoring is designed
to be largely transparent to the programmer but can have large impacts on
program execution time. As a result, the idea of pushing real-time requirements
past the software layer and down into the hardware will be important for future
computer designs. We have shown in this thesis how to handle some of the
challenges as well as how to capture some of the opportunities created from
this co-design of hardware and real-time requirements.
