\chapter{Conclusion}
\label{chap:conclusion}

\section{Summary}

% Summary
%Traditionally, real-time requirements have been handled at the software level,
%with the hardware being agnostic to these timing requirements. However,
%hardware operation can affect timing in ways that are not always exposed to
%software. In addition, opportunities exist for better designs by exposing these
%timing requirements to hardware. 
In this thesis, we have addressed some of the challenges and explored some of
the opportunities for the hardware design of real-time systems. We have looked
at how to design secure, reliable, and energy-efficient real-time systems.

% Security
We have explored how to apply run-time monitoring techniques for improving
security and reliability to real-time systems. We have developed a method
to estimate the worst-case execution time (WCET) of run-time monitoring,
allowing it to be accounted for in real-time scheduling and analysis frameworks
(Chapter~\ref{chap:monitoring_wcet}). Our estimated bounds for run-time
monitoring are within 71\% of observed worst-case performance, similar to the
baseline tools which show up to 52\% differences between simulations and WCET
estimates. We have also developed architectures for applying run-time
monitoring to existing systems with hard real-time deadlines
(Chapter~\ref{chap:monitoring_hard_drop}) or soft performance constraints
(Chapter~\ref{chap:monitoring_dift_drop}). For hard real-time systems, we show
that an average of 15-66\% of monitoring checks, depending on the monitoring
scheme, can be performed with no impact on WCET. Similarly, for
performance-constrained systems, we show average monitoring coverage of 14-85\%
with significantly reduced performance impact compared to performing full
monitoring.

% Energy-efficiency
We have also explored some of the opportunities for reducing the energy usage
of real-time systems. We have shown how soft real-time requirements can be used
to inform the dynamic voltage and frequency scaling of hardware. By developing
a prediction-based DVFS controller, we showed how the appropriate frequency
could be selected for each job in order to minimize energy while meeting its
deadline requirement.  Our experiments showed 56\% average energy savings over
running at maximum frequency by using our prediction-based controller. In
addition, our prediction-based controller outperforms both the Linux
interactive governor and a PID-based controller in energy savings and deadline
misses.

% % Closing thoughts
% As computing systems become more widespread and more deeply embedded in our
% daily lives, their real-time interactions will become increasingly important.
% In this thesis, we have explored the use of modern hardware techniques for
% improving system security, reliability, and energy-efficiency in the context of
% real-time systems. New hardware techniques for improving computer systems are
% continually being proposed, especially as we face new challenges with
% energy-efficiency and the slowing of Moore's Law. In addition, computing
% systems continue to become more deeply in our daily lives. As a result,
% designing hardware with real-time requirements will continue to be an avenue of
% future research.

\section{Future Directions}

\subsection{Secure and Reliable Real-Time Systems}

Run-time monitoring has not yet been implemented in commercial processors.
However, current trends could cause hardware run-time monitoring to be realized
sooner rather than later. Although Moore's Law has continued to increase
transistor count on chips, it is becoming more difficult to find ways to use
these transistors to improve performance. Additionally, the last few years have
seen numerous high-profile and high-cost security attacks occur. 
%Major
%corporations have been attacked and had information leaked. 
The continued
increase in transistor count and the rising importance of computer security
could indicate the beginning of hardware security features, such as run-time
monitoring, in commercial processors in the near future. Applying these
architectures to real-time systems will be especially beneficial, due to the
often safety-critical cyber-physical interactions of real-time systems.  In
addition, these systems have not been immune to attack as seen by recent
reports of attacks in automobiles and airplanes. 

%Run-time monitoring for real-time systems could see more immediate deployment
%in the form of soft-core processors on FPGAs. As this does not require as long
%or expensive a design cycle as ASIC processors, it is easier to justify its
%implementation, especially for niche applications which require high security.

One interesting future research direction is to explore possible gains from
combining static analysis of applications with run-time monitoring. Run-time
monitoring can catch certain errors that would not be caught by static analysis
due to incomplete information (e.g., input-dependent data values). However,
there are a number of properties that can be checked statically. 
By verifying some properties statically and passing this information to the
run-time system, the amount of run-time monitoring performed can be reduced.
This idea is not specific to real-time systems. However, real-time system
design flows already include static analysis for timing guarantees, so
additional analysis for security may fit well in the flow. In addition,
reduction in run-time monitoring can lead to big gains for real-time systems in
terms of reducing WCET and increasing coverage.

\subsection{Energy-Efficient Real-Time Systems}

In this thesis, we have demonstrated energy savings using prediction-based DVFS
control for applications running on a real system.  The main weakness in
extending this DVFS control to commercial applications and systems lies in the
quality of the tools written. We have written a set of tools to instrument code
for features, perform program slicing, and generate the resulting predictor.
The tools written for our experiments may run into issues with more complex
code. However, these tool requirements are not particularly new. For
example, commercial tools exist for performing program slicing, which is the
most complex portion of the toolflow. Thus, the challenges in creating a more
robust toolset are not insurmountable. 

Our work presented in this thesis is a first step in developing general and
automated prediction-based resource control. Generalizing this to other
applications and platforms is a promising direction for future research.
Similar approaches could be applied to DVFS and resource control of other
domains such as GPUs, FPGAs, hardware accelerators, heterogeneous datacenters,
etc. These will require expanding on the prediction methods presented in this
thesis. We have shown one example of this by extending our controller to select
the appropriate core for a heterogeneous system with big and little cores.

On the application side, the workloads we tested worked well with using control
flow features. However, it is likely that irregular workloads exist which do
not show execution time that correlates well with control flow features. This
will require further study into what additional features may be useful for
prediction, such as information about memory patterns. In addition, although we
have avoided needing detailed programmer knowledge, hints from the programmer
have the potential to greatly improve prediction accuracy, especially for
these irregular workloads.

\subsection{Hardware Design for Real-Time Systems}

We have shown in this thesis how to handle some of the challenges as well as
how to capture some of the opportunities created from this co-design of
hardware and real-time requirements.  The increased integration of computers in
our daily lives, especially in the form of cyber-physical systems, is
increasingly creating contexts where real-time is an issue for computing. In
addition, as hardware becomes increasingly complex, it is harder to manage the
timing of tasks purely from the software layer. For example, hardware-based
run-time monitoring is designed to be largely transparent to the programmer but
can have large impacts on program execution time. As a result, the idea of
pushing real-time requirements past the software layer and down into the
hardware will be important for future computer design and research.
