\chapter{Real-Time Systems}
\label{chap:realtime_systems}

\section{Overview}

What are real-time systems. Concept of deadline. Typical model (set of tasks
with deadlines).

Why these are important.  Applications (i.e, control systems such as
automobiles, planes, medical systems, etc.)

\section{Scheduling}

The scheduling problem is given a set of real-time tasks with activation times,
deadlines, which task should be run at any given time in order to meet all
deadlines. A host of work has done on real-time scheduling algorithms to handle
things such as increase CPU utilization, periodic and aperiodic tasks, etc,
fixed vs. varying priority.

Two traditional schedulers are rate-monotonic scheduling and earliest deadline first.

\subsection{Rate-Monotonic Scheduling}

\subsection{Earliest-Deadline First Scheduling}

\section{Worst-Case Execution Time}

One important component needed for most real-time schedulers is the worst-case
execution time of tasks. This is used in order to make scheduling decisions and
to provide strong guarantees that deadlines will be met. Worst-case execution
time is the longest execution time of a program running on a particular system
under all possible inputs and architectural states. Oftentimes, techniques such
as cache flushing are used to provide a consistent initial architectural state
in order to simplify WCET analysis. WCET analysis is typically done with static
analysis techniques because of the state explosion problem of attempting to run
or simulate all possible inputs. Recent work has looked into statical sampling
in order to provide a (non-conservative) WCET estimate. However, conservative
estimates still require static analysis. The typical method for doing this is
based on the implicit path-enumeration technique (IPET).

\subsection{Implicit-Path Enumeration Technique}

IPET estimates the WCET of a task by constructing an integer-linear programming
(ILP) problem and finding the maximum of this problem.
